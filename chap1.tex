\chapter{这是第一章的标题}
%\AddToShipoutPicture*{\BackgroundPic{OWT-chap1.pdf}}
[按章节逐一粘贴/键入论文内容,粘贴/键入第1章正文内容]
%\noindent\phantom{空格}%
正文是论文的核心部分,占主要篇幅,论文的论点、论据和论证都在这里阐述。由于论文作者的研究工作涉及的学科、研究对象和研究方法和结果表达方式等差异很大,所以对正文的撰写内容不作统一规定。但总的思路和结构安排应当符合“提出论点,通过论据或数据对论点加以论证”这一共同的要求。正文应达到观点正确,结构完整、合乎逻辑、符合学术规范,无重大疏漏或明显的片面性。其他具体要求有:

(1)主题的要求

A.	主题有新意,有科学研究或实际应用价值;

B.	主题集中,一篇论文只有一个中心,要使主题集中,凡与本文主题无关或关系不大的内容不应涉及,不过多阐述,否则会使问题繁杂,脉络不清,主题淡化;

C.	主题鲜明,论文的中心思想地位突出,除了在论文的题目、摘要、前言、结论部分明确地点出主题外,在正文部分更要注意突出主题。

(2)结构的要求

A.	不同内容的正文,应灵活处理,采用合适的结构顺序和结构层次,组织好段落,安排好材料。 章、节、小节等分别以“1”、“1.1”、“1.1.1”、“1.1.2”、“(1)”或“一、”、“(一)”、“1.”、“(1)”等数字以树层次格式依次标出。

B.	正文写作时要注意抓住基本观点。数据的采集、记录、整理、表达等均不应出现技术性的错误;分析论证和讨论问题时,避免含混不清,模棱两可,词不达意 ;不弄虚作假。

\section{节标题}
%\noindent\phantom{空格}%
\zhlipsum[2-3]
\subsection{$a^2+b^2=c^2$}
%\noindent\phantom{空格}%
\zhlipsum[5]

\section{title}
%\noindent\phantom{空格}%
\zhlipsum[8]