%abs-chinese
\newpage
\clearpage
\cleardoublepage
\setcounter{page}{1}
%\phantomsection\addcontentsline{toc}{chapter}{摘要}%把摘要加入目录
%\AddToShipoutPicture*{\BackgroundPic{OWT-abs-chinese.pdf}}
\begin{center}
\vspace*{1.3mm}
\heiti\fontsize{16}{16}\textbf{\mytitle}\\
\vspace{11mm}
\fontsize{15}{15}
\textbf{摘}\hspace{5.3mm}\textbf{要}
\end{center}
\vspace{4.5mm}
\fontsize{14}{20}\selectfont
\noindent\phantom{空格}%修改你的中文摘要:
[请点击此处,然后粘贴/输入中文摘要内容]

%摘要是对论文内容不加注释和评论的简明归纳,应包括研究工作的目的、方法和结论,重点是结果和结论。用语要规范,一般不用公式和非规范符号术语,不出现图、表、化学结构式等。采用第三人称撰写,一般在300字左右。

%关键词是为了满足文献标引或检索工作的需要而从论文中选取出的用以表示全文主题内容信息的词或词组。关键词包括主题和自由词:主题词是专门为文献的标引或检索而从自然语言的主要词汇中挑选出来并加以规范化了的词或词组;自由词则是未规范化的即还未收入主题词表中的词或词组。

%每篇论文中应列出3~8个关键词,它们应能反映论文的主题内容。其中主题词应尽可能多一些,关键词作为论文的一个组成部分,列于摘要段之后。还应列出与中文对应的英文关键词(Key words)。关键词间空格1个字符。


\vspace{11.5mm}%这一行前后的要留空白行!

\noindent\phantom{空格}\heiti\textbf{%修改你的关键词
	关键词:\LaTeX\ 模板
}