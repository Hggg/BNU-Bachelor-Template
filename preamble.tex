%preamble
\documentclass[12pt,a4paper,openany,oneside]{book}
%-------------------------------------------------------------------------------
%package
\usepackage{amsmath,amssymb,amsthm}

\usepackage[AutoFakeBold=2.5]{xeCJK}
\usepackage{indentfirst}
\usepackage{newtxtext}%times字体
\usepackage[slantedGreek]{newtxmath}

\usepackage[top=2.5cm, bottom=2.5cm, left=2.5cm, right=2cm]{geometry}

\usepackage{fancyhdr}
\pagestyle{fancyplain}
\fancyhead{}
\fancyfoot[L,R]{}
\fancyfoot[C]{\vspace*{-6mm}\fontsize{9}{9}\selectfont\thepage}
\renewcommand{\headrulewidth}{0pt}

\usepackage{setspace}
\usepackage[T1]{fontenc}
\usepackage{pict2e}
\usepackage{color}
\usepackage{graphicx,float}
\usepackage{lipsum,zhlipsum}%生成随机文本,可以去掉

%加上你自己的package
\usepackage{}

%参考文献
%Use EITHER 1
\usepackage[noadjust]{cite}
%OR 2:
%%meet the standard for bib: GB/T 7714-2015 
%\usepackage[backend=biber,style=gb7714-2015]{biblatex} 
%\addbibresource{bibfile}


%hyperref
%\usepackage[notref,notcite]{showkeys}%option=notref,notcite
\usepackage[hidelinks]{hyperref}
%\hypersetup{%修改设置
%	colorlinks=true,%
%	pdfborder= 0 0 0 ,%
%	%pdfpagemode=FullScreen,%option=UseNone, UseThumbs(show thumbnails), UseOutlines (show book-marks), FullScreen, UseOC (PDF 1.5), and UseAttachments
%	pdftitle={title},%
%	pdfsubject={subject},%
%	pdfauthor={author},%
%	pdfkeywords={keyword},%
%}
\usepackage{cleveref}
%-------------------------------------------------------------------------------
%自定义
%macro
%\def\dd{\mathrm{d}}
%\numberwithin{equation}{section}
%\newcommand\abs[1]{\left\lvert #1 \right\rvert}
%\newcommand\norm[1]{\left\lVert #1 \right\rVert}
%\let\div\rest
%\DeclareMathOperator{\div}{div}
%\def\dis{\displaystyle}
%
%%environment
%\newtheorem{lemma}{Lemma}[chapter]
%\newtheorem{theorem}[lemma]{Theorem}
%\newtheorem{corollary}[lemma]{Corollary}
%\theoremstyle{remark}
%\newtheorem*{remark}{Remark}
%\newtheorem*{solution}{Solution}
%\theoremstyle{definition}
%\newtheorem{definition}{Definition}[chapter]
%
%

%-------------------------------------------------------------------------------
%各种设置,慎改!
%font
\setCJKmainfont{SimSun}
\setCJKsansfont{SimHei}
\setCJKmonofont{FangSong}
\setCJKfamilyfont{song}{SimSun}
\setCJKfamilyfont{hei}{SimHei}
\setCJKfamilyfont{kai}{KaiTi}
\setCJKfamilyfont{fs}{FangSong}
\newcommand{\song}{\CJKfamily{song}}    % 宋体
\def\songti{\song}
\newcommand{\fs}{\CJKfamily{fs}}        % 仿宋体
\def\fangsong{\fs}
\newcommand{\kai}{\CJKfamily{kai}}      % 楷体
\def\kaishu{\kai}
\newcommand{\hei}{\CJKfamily{hei}}      % 黑体
\def\heiti{\hei}

%toc
\renewcommand{\contentsname}{\vspace*{9.8mm}\fontsize{16}{16}\heiti\textbf{目}\hspace{5.8mm}\textbf{录}}
%如果需要把“目录”二字改为英文,注释掉上一行,并取消注释下一行
%\renewcommand{\contentsname}{\vspace*{9.8mm}\fontsize{16}{16}Contents}
%
%\usepackage{tocbibind}%把目录、参考文献加入目录%和当前文档中的几条命令重复
\usepackage[titles]{tocloft}
\setcounter{secnumdepth}{2}
\setcounter{tocdepth}{2}
%If you want SimHei for number
%\renewcommand{\cftchappagefont}{\heiti\setmainfont{SimHei}}
\renewcommand{\cftchappagefont}{}
\renewcommand{\cftdot}{…}
\renewcommand{\cftdotsep}{0}
\renewcommand{\cftchapleader}{\cftdotfill{\cftdotsep}}
\renewcommand{\cftsubsecleader}{\cftdotfill{\cftdotsep}}
\renewcommand{\cftsecindent}{\cftchapindent}
\renewcommand{\cftsubsecindent}{\cftchapindent}
\renewcommand{\cftchapfont}{\fontsize{16}{24}\selectfont\heiti}
\renewcommand{\cftsecfont}{\fontsize{15}{22.5}\selectfont\heiti}
\renewcommand{\cftsubsecfont}{\fontsize{14}{21}\selectfont\heiti}
%规范中未对目录不同项之间的间距做规定,以下数值取半倍行高
\renewcommand{\cftbeforechapskip}{8pt}
\renewcommand{\cftbeforesecskip}{7.5pt}
\renewcommand{\cftbeforesubsecskip}{7pt}

%title
\usepackage{titlesec}
\titleformat{\chapter}{\vspace*{-19.8mm}\centering\fontsize{16}{16}\bfseries\heiti}{\thechapter}{1em}{\vspace*{-9.5mm}}
\titleformat{\section}{\vspace*{-0.2mm}\fontsize{15}{15}\bfseries\heiti}{\thesection}{1em}{\vspace{1.1mm}}
\titleformat{\subsection}{\vspace*{0.6mm}\fontsize{14}{14}\bfseries\heiti}{\thesubsection}{1em}{\vspace{2.6mm}}
\renewcommand\bibname{\vspace*{9mm} 参考文献\vspace{-9mm}}

%封面页横线
\makeatletter
\newlength\ae@tmp@length
\newcommand\aeunderline[2][]{%%
	\settowidth\ae@tmp@length{#2}%%
	\makebox[0pt][r]{#2}%%
	\hspace{-\ae@tmp@length}%%
	\rule[\dimexpr-0.25ex-4pt]{\dimexpr\ae@tmp@length+#1+0pt\relax}{1pt}}
\makeatother

\newcommand*{\ziju}[1]{\renewcommand{\CJKglue}{\hskip #1}}
\usepackage[absolute,overlay]{textpos}

%修改\cleardoublepage的定义
%ref: https://tex.stackexchange.com/questions/185821/openright-in-oneside-book
\makeatletter
\def\cleardoublepage{\clearpage\ifodd\c@page\else
	\hbox{}\newpage\if@twocolumn\hbox{}\newpage\fi\fi}
\makeatother
