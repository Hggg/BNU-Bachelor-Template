%appendix
%\AddToShipoutPicture*{\BackgroundPic{OWT-appendix.pdf}}
\fontsize{12}{20}\selectfont
\phantomsection\addcontentsline{toc}{chapter}{附录}
\begin{center}
	\vspace*{-2mm}
	\heiti\fontsize{16}{16}\selectfont\textbf{附\hspace{5.7mm}录}
\end{center}

论文的附录依序编排为附录 1,附录 2...。附录中的图表公式另编排序号,与正文分开。

附录是论文主体的补充项目,为了体现整篇论文的完整性,写入正文又可能有损于论文的条理性、逻辑性和精炼性,这些材料可以写入附录段,但对于每一篇论文并不是必须的。主要包括以下几类:

(1)比正文更为详尽的理论根据、研究方法和技术要点,建议可以阅读的参考文献的题录,对了解正文内容有用的补充信息等;

(2)由于篇幅过长或取材于复制品而不宜写入正文的材料;

(3)一般读者并非必要阅读,但对本专业同行很有参考价值的资料;

(4)某些重要的原始数据、数学推导、计算程序、框图、结构图、统计表、计算机打印输出件等。
附录段置于参考文献表之后,附录中的插图、表格、公式、参考文献等的序号与正文分开,另行编制,如编为“图一”、“图二”;“表一”、“表二”;“式(一)”、“式(二)”;“文献[一]”、“文献[二]”等。
